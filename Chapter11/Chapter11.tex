%!TEX program = xelatex
\documentclass[UTF8,aspectratio=169,10pt,t]{ctexbeamer}

\mode<presentation> {
\usetheme{Madrid}
\setbeamertemplate{footline}[frame number]
\setbeamercolor{page number in head/foot}{fg=blue} %设置页码颜色
\setbeamertemplate{navigation symbols}{}
}
\usepackage{indentfirst}
\setlength{\parindent}{2em}
\usepackage{listings}
\usepackage{wrapfig}
\usepackage{graphicx}

%自定义块
\usepackage{setspace}
\definecolor{hanblue}{rgb}{0.27, 0.42, 0.81}
\definecolor{indiagreen}{rgb}{0.07, 0.53, 0.03}
\definecolor{indianred}{rgb}{0.8, 0.36, 0.36}
\definecolor{indianyellow}{rgb}{0.89, 0.66, 0.34}
\definecolor{babypink}{rgb}{0.96, 0.76, 0.76}
\definecolor{ao(english)}{rgb}{0.0, 0.5, 0.0}
% \setbeamerfont{block title}{size=\small}
% \setbeamerfont{block body}{size=\small}
\setbeamerfont{block title}{}
\setbeamerfont{block body}{}
\newenvironment<>{blueblock}[1]{
\setbeamercolor{block title}{fg=white,bg=hanblue}
\begin{block}#2{#1}}{\end{block}}
\newenvironment<>{greenblock}[1]{
\setstretch{1.3}\setbeamercolor{block title}{fg=white,bg=indiagreen}
\begin{block}#2{#1}}{\end{block}}
\newenvironment<>{redblock}[1]{
\setstretch{1.3}\setbeamercolor{block title}{fg=white,bg=indianred}
\begin{block}#2{#1}}{\end{block}}
\newenvironment<>{yellowblock}[1]{
\setstretch{1.3}\setbeamercolor{block title}{fg=white,bg=indianyellow}
\begin{block}#2{#1}}{\end{block}}

\lstset{language=C++,
columns=flexible,
basicstyle=\footnotesize\ttfamily,                                      % 设定代码字体、大小
%numbers=left,xleftmargin=2em,framexleftmargin=2em,                   % 在左侧显示行号
%numberstyle=\color{darkgray},                                        % 设定行号格式
keywordstyle=\color{blue},                                            % 设定关键字格式
commentstyle=\color{ao(english)},                                     % 设置代码注释的格式
stringstyle=\color{brown},                                            % 设置字符串格式
%showstringspaces=false,                                              % 控制是否显示空格
%frame=lines,                                                         % 控制外框
breaklines,                                                           % 控制是否折行
postbreak=\space,                                                     % 控制折行后显示的标识字符
breakindent=5pt,                                                      % 控制折行后缩进数量
emph={size\_t,array,deque,list,map,queue,set,stack,vector,string,pair,tuple,function,placeholders,std,nullptr,unique\_ptr}, % 非内置类型
emphstyle={\color{teal}},
escapeinside={(*@}{@*)},
%调整马德里的代码行号
% numbers=left,
% xleftmargin=2em,
% framexleftmargin=2em
}

\title[\textit{C++程序设计:第十一章}]{第十一章~标准模板库}

%\author[李长河]{李长河} % Your name
%\institute[CUG] % Your institution as it will appear on the bottom of every slide, may be shorthand to save space
%{
%中国地质大学(武汉)\\ % Your institution for the title page
%\medskip
%\textit{lichanghe@cug.edu.cn} % Your email address
%}
\date{} % Date, can be changed to a custom date


\begin{document}

\maketitle

\begin{frame}
	{目录}
	\tableofcontents
\end{frame}

\begin{frame} {前言}
	\begin{yellowblock}{学习目标}
		\begin{itemize}
			\item 理解迭代器的工作原理和使用方法;
			\item 理解常见容器的特点并掌握它们的使用方法;
			\item 了解算法的类型并掌握常用调用对象的使用方法。
		\end{itemize}
	\end{yellowblock}
\end{frame}

\begin{frame} {前言}
	\begin{block}
		{标准模板库}
		标准模板库(standard template library,STL)是C++标准库(standard library)的重要组成部分,其包含以下几个部分:
		\begin{description}
			\item [容器(container)]常用的数据结构,包括vector、list等
			\item [算法(algorithm)]操作容器的泛型算法,包括查找、排序等
			\item [迭代器(iterator)]容器和算法之间的桥梁,处理不同类型容器的途径
		\end{description}
	\end{block}
\end{frame}

\section{迭代器}
\begin{frame}{11.1~迭代器}
	我们已经在4.6.3节介绍了\textbf{迭代器(iterator)},并使用它访问vector类型中的元素。本节我们考虑以下问题:
	\begin{greenblock}{问题引入}
		给定一个vector或者数组以及一个数据值,要求\alert{查找给定的数据值是否在vector或者数组中}:
		\begin{itemize}
			\item 找到的话,返回该元素的地址
			\item 没有找到,返回一个空指针
		\end{itemize}
	\end{greenblock}
\end{frame}

\subsection{实现Find函数模板}
\begin{frame}[fragile]{11.1~迭代器\small{——实现Find函数模板}}
	根据问题要求,vector类型和数组类型的查找算法函数模板如下:
	\begin{columns}[t]
		\column{0.65\textwidth}
		\begin{blueblock}{查找函数模板}
			\vspace{-3mm}\begin{lstlisting}[basicstyle=\small\ttfamily]
template<typename T>
const T* Find(const vector<T> &vec, const T &val) {
    for (int i=0; i<vec.size(); ++i)
        if (vec[i] == val) return &vec[i];
    return nullptr;
}
template<typename T>
const T* Find(const T *arr, int size,const T &val) {
    if (!arr || size <= 0)
        return nullptr;
    for (int i = 0; i<size; ++i)
        if (arr[i] == val) return &arr[i];
    return nullptr;
}
            \end{lstlisting}
		\end{blueblock}
		\column{0.3\textwidth}
		\begin{greenblock}
			{观察}
			两个函数体中循环部分代码是否一样?\\
		\end{greenblock}
		\pause
		\begin{greenblock}
			{问题}
			那么是否可以通过同一个Find函数来处理vector和数组呢?\\
		\end{greenblock}
	\end{columns}
\end{frame}

\begin{frame}[fragile]{11.1~迭代器}{实现Find函数模板}
	给定线性结构\alert{第一个和尾后元素的地址},通过指针访问之,就可以实现一个通用算法如下:
	\begin{blueblock}{通用算法}
		\begin{lstlisting}[basicstyle=\small\ttfamily]
template<typename T>
const T* Find(const T *first, const T *last, const T &val) {
    if (!first || !last )
        return nullptr;
    for (;first!=last;++first)//last为尾后元素的地址
        if (*first == val)
            return first;
    return nullptr;
}
        \end{lstlisting}
	\end{blueblock}
\end{frame}

\begin{frame}[fragile]{11.1~迭代器\small{——实现Find函数模板}}
	测试代码如下:
	\begin{blueblock}{通用算法}
		\begin{lstlisting}[basicstyle=\small\ttfamily]
if (auto p = Find(arr, arr + sizeof(arr) / sizeof(int), 5))
    cout << *p << endl; //处理数组
if (auto p = Find(&vi[0], &vi[vi.size() - 1]+1, 4))
    cout << *p << endl; //处理vector
        \end{lstlisting}
	\end{blueblock}
	\begin{greenblock}{思考}
		以上代码需要用户自行指定第一个和尾后元素的地址,比较麻烦。\\
		我们是否可以简化这个过程?
	\end{greenblock}
\end{frame}

\begin{frame}[fragile]{11.1~迭代器\small{——实现Find函数模板}}
	将vector的第一个元素和尾后元素取地址的操作包装为如下的\textbf{Begin和End函数模板}:
	\begin{blueblock}{Begin和End函数模板(vector)}
		\begin{lstlisting}[basicstyle=\small\ttfamily]
template<typename T>
const T* Begin(const vector<T> &vec) {
    return vec.size() > 0 ? &vec[0] : nullptr;
}
template<typename T>
const T* End(const vector<T> &vec) {
    return vec.size() > 0 ? &vec[vec.size()-1]+1 : nullptr;
}
        \end{lstlisting}
	\end{blueblock}
\end{frame}

\begin{frame}[fragile]{11.1~迭代器\small{——实现Find函数模板}}
	对数组首元素和尾后元素的封装类似:
	\begin{blueblock}{Begin和End函数模板(数组)}
		\begin{lstlisting}[basicstyle=\small\ttfamily]
template<typename T, size_t N>
const T* Begin(const T (&arr)[N]) {
    return arr;
}
template<typename T, size_t N>
const T* End(const T (&arr)[N]) {
    return arr + N;
}
        \end{lstlisting}
    \end{blueblock}
    其中,模板函数形参arr是实参数组的引用。
\end{frame}

\begin{frame}[fragile]{11.1~迭代器\small{——实现Find函数模板}}
	使用封装后的Begin和End调用Find函数的示例如下:
	\begin{blueblock}{传入Begin和End调用Find函数}
		\begin{lstlisting}[basicstyle=\small\ttfamily]
Find(Begin(vi), End(vi), 4);
Find(Begin(arr), End(arr), 4);
        \end{lstlisting}
	\end{blueblock}
	\pause
	和原始调用方式比较:
	\begin{blueblock}{原始调用方式}
		\begin{lstlisting}[basicstyle=\small\ttfamily]
Find(&vi[0], &vi[vi.size() - 1]+1, 4) //处理vector
Find(arr, arr + sizeof(arr) / sizeof(int), 4) //处理数组
    \end{lstlisting}
	\end{blueblock}
	可见简洁很多。
\end{frame}

\subsection{使用迭代器}
\begin{frame}[fragile]{11.1~迭代器\small{——使用迭代器}}
	每一种容器都有一个与之关联的迭代器。可以通过成员函数begin和end获取第一个元素和尾后元素的迭代器,如:
	\begin{blueblock}{使用迭代器}
		\begin{lstlisting}[basicstyle=\small\ttfamily]
vector<int> vi = {0,1,2,3};
vector<int>::iterator itb = vi.begin(); //itb指向vi 的首元素
vector<int>::iterator ite = vi.end(); //ite指向vi的尾后元素
        \end{lstlisting}
	\end{blueblock}
	定义一个迭代器对象时,我们必须要指明与之关联的容器和元素类型。如上述代码中迭代器指向\texttt{vector<int>}中的元素。\\
	\pause
	通常我们会用auto来简化迭代器定义
	\begin{blueblock}{利用auto简化迭代器定义}
		\begin{lstlisting}[basicstyle=\small\ttfamily]
auto itb = vi.begin(); //利用auto简化定义
        \end{lstlisting}
	\end{blueblock}
\end{frame}

\begin{frame}[fragile]{11.1~迭代器\small{——使用迭代器}}
	无论指向何种容器,迭代器都支持以下操作:
	\begin{itemize}
		\item \textbf{解引用与成员选择}:*iter, iter->member(等价于(*iter).member);
		\item \textbf{自增运算符}:++iter, iter++;
		\item \textbf{赋值运算符}:iter1 = iter2;
		\item \textbf{关系 == 和 != 运算}:iter1 == iter2, iter1 != iter2。\\
	\end{itemize}
	\pause
	\vspace{0.5cm}
	通常,我们使用迭代器来遍历容器中的元素
	\begin{blueblock}
		{使用迭代器进行遍历}
		\begin{lstlisting}[basicstyle=\small\ttfamily]
for(auto it = vi.begin(); it != vi.end(); ++it){ //遍历vector
    cout << *it <<endl;
}
            \end{lstlisting}
	\end{blueblock}
	几乎STL提供的所有算法都是通过迭代器实现对容器中元素的操作,即通过接受由begin和end划定的\alert{左闭合区间[begin,end)},对区间内元素进行操作。
\end{frame}

\begin{frame}[fragile]{11.1~迭代器\small{——使用迭代器*}}
	迭代器的简单分类如下:
	\begin{itemize}
		\item \textbf{输入(input)迭代器}:只能单步向前迭代(自增运算++),\alert{不允许修改由该类迭代器引用的元素};
		\item \textbf{输出(output)迭代器}:该类迭代器和输入迭代器相似,也只能单步向前迭代,不同的是该类迭代器对引用的元素\alert{只能执行写操作};
		\item \textbf{前向(forward)迭代器}:该类迭代器可以在一个正确的区间中进行读写操作,它拥有输入和输出迭代器的特性,\alert{仅支持自增运算};
		\item \textbf{双向(bidirectional)迭代器}:该类迭代器是在前向迭代器基础上提供了单步向后迭代的功能,\alert{支持自增(++)和自减(--)运算};
		\item \textbf{随机访问(random access)迭代器}:该类迭代器具有上面所有迭代器的功能,并能\alert{直接访问容器中任意一个元素},支持iter+n, iter-n, iter+=n, iter-=n,iter1-iter2。
	\end{itemize}
	\begin{redblock}{注意}
		一个迭代器的类型取决于\alert{与其关联的容器类型},比如一个指向vector类型的迭代器的类型为随机访问迭代器。
	\end{redblock}
\end{frame}

\section{容器}
\subsection{容器概述}
\begin{frame}{11.2~容器\small{——容器概述}}
	\begin{block}
		{容器}
		容器是\alert{特定类型对象的集合},集合中的元素通过某种数据结构组织在一起
	\end{block}
	STL中的容器主要由两大类组成:
	\begin{itemize}
		\item 顺序容器
		\item 关联容器
	\end{itemize}
	\vspace{0.5cm}
\pause
	\par
	大多数容器都支持以下操作:
	\begin{itemize}
		\item 关系运算
		\item 赋值运算成员
		\item begin和end成员
		\item empty成员
		\item size成员
		\item clear成员
	\end{itemize}
\end{frame}

\begin{frame}[fragile]{11.2~容器\small{——容器概述*}}
	一般而言,每个容器都定义在一个与容器名相同名称的头文件中,使用时包含之即可
	\begin{blueblock}{包含容器头文件}
		\begin{lstlisting}[basicstyle=\small\ttfamily]
#include <vector>
                \end{lstlisting}
	\end{blueblock}
	每一种容器均被定义为类模板,因此在使用时需要提供额外的模板参数信息
	\begin{blueblock}{提供容器模板参数}
		\begin{lstlisting}[basicstyle=\small\ttfamily]
vector<string> vs;
            \end{lstlisting}
	\end{blueblock}
\end{frame}

\begin{frame}[fragile]{11.2~容器\small{——容器概述*}}
	C++11 还为每一种容器提供了\textbf{cbegin和cend成员},分别返回第一个元素和尾后元素的
	const迭代器,\alert{不允许对指向的元素执行写操作}
	\begin{blueblock}{cbegin和cend}
		\begin{lstlisting}[basicstyle=\small\ttfamily]
vector<int> vec = {0, 1, 2, 3, 4};
auto it1 = vec.begin(); //返回第一个元素的迭代器
auto it2 = vec.cbegin(); //返回第一个元素的const迭代器
*it1 = 4; //正确:修改第一个元素的值
*it2 = 5; //错误:it2为const 迭代器,不允许修改指向的对象
                \end{lstlisting}
	\end{blueblock}
	\begin{yellowblock}
		{说明}
		对于begin成员,只有当容器是const类型,才返回const类型迭代器;否则返回非const。为了避免不必要的修改错误,C++增加了上述cbegin和cend成员。
	\end{yellowblock}
\end{frame}

\begin{frame}[fragile]{11.2~容器\small{——容器概述*}}
	以下代码展示vector的插入和删除数据操作:
	\begin{columns}
		\column{0.65\textwidth}
		\begin{blueblock}{vector的插入和删除数据操作}
			\begin{lstlisting}[basicstyle=\small\ttfamily]
vector<int> vec = {0, 1, 2, 3, 4};
vec.insert(vec.begin(),10); //在首部插入10
vec.erase(vec.begin()+1); //删除vec中第二个元素
                    \end{lstlisting}
		\end{blueblock}
		\column{0.3\textwidth}
		\begin{redblock}{注意}
			除C++11新增的array 容器以外,其它容器都是可变长的
		\end{redblock}
	\end{columns}
\end{frame}

%!这里有问题!
\begin{frame}[fragile]{11.2~容器\small{——容器概述}}
	C++11为可变长容器新增的\textbf{emplace成员},示例如下:
	\begin{columns}
		\column{0.65\textwidth}
		\begin{blueblock}{定义一个foo类}
			\begin{lstlisting}[moreemph={T}]
struct Foo{
    Foo(const string &name, int id) :m_name(name), m_id(id) {}
    string m_name;
    int m_id;
};
                    \end{lstlisting}
		\end{blueblock}
		\column{0.3\textwidth}
	\end{columns}
\end{frame}

\begin{frame}[fragile]{11.2~容器\small{——容器概述}}
	C++11为可变长容器新增的\textbf{emplace成员},示例如下:
	\begin{blueblock}{emplace与insert成员用法区别}
		\begin{lstlisting}[moreemph={T}]
vector<Foo> vf;
vf.push_back(Foo("Lisha", 12)); //将临时对象移到容器的末尾
vf.insert(vf.begin(),Foo("Mandy", 13)); //将临时对象移到容器的开始位置
vf.emplace_back("Kevin", 11); //在容器的末尾新增一个元素
vf.emplace(vf.begin(),"Rosieta", 10); //在容器的首部插入一个元素
                \end{lstlisting}
	\end{blueblock}
\pause
	\begin{yellowblock}{说明}
		$\bullet$ emplace成员通过一个\alert{参数包}接受的参数来\alert{构造一个元素}并将之插入容器中。\\
		$\bullet$ emplace\_back函数调用把两个实参传递给Foo类的构造函数,并在vf的末尾\alert{利用这两个参数值构造一个新元素}\\
		$\bullet$ 与emplace\_back成员相比,push\_back和insert成员只能移动或复制\alert{已构造元素}\\
	\end{yellowblock}
\end{frame}

\begin{frame}[fragile]{11.2~容器\small{——容器概述*}}
	\textbf{swap操作}交换两个相同类型容器的数据:
	\begin{columns}
		\column{0.65\textwidth}
		\begin{blueblock}{swap操作}
			\begin{lstlisting}[moreemph={T}]
vector<int> v1 = {0, 1}; //2个元素的vector
vector<int> v2 = {0, 1, 2, 3}; //4个元素的vector
swap(v1, v2);
                    \end{lstlisting}
		\end{blueblock}
		\column{0.3\textwidth}
		\begin{yellowblock}{说明}
			$\bullet$ 调用swap 函数之后,v1将包含4个元素,v2将包含2个元素。\\
		\end{yellowblock}
	\end{columns}
	\begin{redblock}{注意}
		$\bullet$ 除array之外,swap函数不会执行任何数据复制、插入或删除操作。\\
		$\bullet$ 对于array来说,swap会真正交换相同位置的元素。
	\end{redblock}
	\begin{greenblock}{思考}
		对于array和其他容器而言,swap操作之后与容器绑定的迭代器、指针以及所指向的元素是否发生了变化?各发生了怎样的变化?
	\end{greenblock}
\end{frame}

\subsection{顺序容器}
\begin{frame}[fragile]{11.2~容器\small{——顺序容器}}
	\begin{block}{顺序容器}
		顺序容器都是线性结构,提供了元素的快速顺序访问能力。
	\end{block}
	\begin{yellowblock}{注意}
		但对于非线性访问和元素增减操作,它们有很大的性能差别。
		\begin{itemize}
			\item 与vector、string、deque、array等容器绑定的迭代器支持随机访问
			\item 与list绑定的迭代器支持双向单步迭代
			\item 与forward\_list绑定的迭代器只支持前向单步迭代。
		\end{itemize}
	\end{yellowblock}
\end{frame}

\begin{frame}[fragile]{11.2~容器\small{——顺序容器}}
	定长数组容器array的使用如下:
	\begin{columns}
		\column{0.65\textwidth}
		\begin{blueblock}{使用array}
			\begin{lstlisting}[moreemph={T}]
array<int, 4> arr = {1,2,3,4};
for (auto it = arr.begin(); it != arr.end(); ++it)
cout << *it << endl;
array<int, 4> arr2 = arr; //array对象允许复制
arr2.fill(0); //所有元素赋值为0
                    \end{lstlisting}
		\end{blueblock}
		\column{0.3\textwidth}
		\begin{yellowblock}{说明}
			$\bullet$ 相比于普通数组,array更安全、易使用\\
			$\bullet$ array不支持插入、删除等改变容器大小的操作\\
			$\bullet$ 但array对象支持赋值和复制操作,还能通过size成员获取数组的大小\\
		\end{yellowblock}
	\end{columns}
\end{frame}

\begin{frame}[fragile]{11.2~容器\small{——顺序容器}}
	双端队列容器deque的使用如下:
	\begin{columns}
		\column{0.65\textwidth}
		\begin{blueblock}{使用deque}
			\begin{lstlisting}[moreemph={T}]
deque<int> dq = {1,2,3};
dq.push_back(4); //尾部插入一个元素
dq.push_front(0); //首部插入一个元素
cout << dq[3]<< endl; //随机访问
            \end{lstlisting}
		\end{blueblock}
		\column{0.3\textwidth}
		\begin{yellowblock}{说明}
			$\bullet$ 与vector类似,deque支持随机访问\\
			$\bullet$ 与vector不同的是,deque可以在\alert{首尾两端}进行快速地插入和删除操作\\
		\end{yellowblock}
	\end{columns}
	\begin{redblock}{注意}
		deque的随机访问效率比容器要低很多。\\
		因其由一些在内存中\alert{互相独立}的动态数组组成\\
	\end{redblock}
\end{frame}

\begin{frame}[fragile]{11.2~容器\small{——顺序容器*}}
	forward\_list的使用如下:
	\begin{columns}
		\column{0.65\textwidth}
		\begin{blueblock}{使用forward\_list}
			\begin{lstlisting}[moreemph={T}]
forward_list<int> flst = { 2, 3 };
flst.push_front(1); //在flst首部插入数据
flst.insert_after(flst.before_begin(), 0); //同上
for (auto it = flst.begin(); it != flst.end(); ++it)
    cout << *it << " "; //打印输出:0 1 2 3
            \end{lstlisting}
		\end{blueblock}
		要获取forward\_list中的元素数目,可以使用distance函数:
		\begin{blueblock}{使用distance获取forward\_list元素数目}
			\begin{lstlisting}[moreemph={T}]
cout << "size: " <<
  distance(flst.begin(), flst.end()) << endl;
            \end{lstlisting}
		\end{blueblock}
		\column{0.3\textwidth}
		\only<1>{\begin{yellowblock}{说明}
				$\bullet$ forward\_list仅提供给定位置\alert{后}的插入删除操作\\
				$\bullet$ \alert{before\_begin成员}返回的迭代器是不能解引用的\\
				$\bullet$ 出于性能考虑,forward\_list\alert{放弃了size函数}\\
			\end{yellowblock}}
		\only<2->{\begin{redblock}{注意}
				与vector、array、deque相比,forward\_list\alert{不支持随机访问}。但对于元素的\alert{插入、删除、移动}等操作,它的性能要好于前三者\\
			\end{redblock}}
	\end{columns}
\end{frame}

\begin{frame}[fragile]{11.2~容器\small{——顺序容器}}
	list的使用如下:
	\begin{blueblock}{使用list}
		\vspace{-3mm}\begin{lstlisting}[moreemph={T}]
list<int> lst1 = {2,3}, lst2 = {1};
lst1.push_back(5); //在lst1的尾部插入元素5
lst2.push_front(0); //在lst2的首部插入元素0
auto pos = find(lst1.begin(),lst1.end(),5);//找到指向元素5的迭代器
lst1.insert(pos, 4);//在此位置插入元素4
lst1.splice(lst1.begin(), lst2); //将lst2插入到lst1中第1个元素位置
for (auto it = lst1.begin(); it != lst1.end(); ++it)
    cout << *it << " "; //打印输出:0 1 2 3 4 5
            \end{lstlisting}\vspace{-2.5mm}
	\end{blueblock}
	\begin{yellowblock}{说明}
		$\bullet$ list是一个\alert{双向链表}\\
		$\bullet$ 除了insert、push\_back等成员可以执行插入操作外,可以使用splice成员将\alert{一个list中的元素转移到另外一个list中}\\
		$\bullet$ splice函数调用执行完后,lst2变为空列表\\
	\end{yellowblock}
\end{frame}

\begin{frame}[fragile]{11.2~容器\small{——顺序容器}}
	list的使用如下:
	\begin{columns}
		\column{0.65\textwidth}
		\begin{blueblock}{使用splice移动某一范围元素}
			\begin{lstlisting}[moreemph={T}]
list<int> lst3 = { 6,7,8 };
auto it = lst3.begin();
//将it后移两个位置
advance(it , 2);
//将lst3中前两个元素转移到lst1的尾部
lst1.splice(lst1.end(), lst3, lst3.begin(),it);
            \end{lstlisting}
		\end{blueblock}
		\column{0.3\textwidth}
		\begin{yellowblock}{说明}
			$\bullet$ splice
			调用将lst3 中从begin到it范围内的元素转移到lst1的尾部\\
			$\bullet$ 上面splice调用结束时,lst1的尾部新增两个元素,lst3剩余1个元素。\\
		\end{yellowblock}
	\end{columns}
\end{frame}

\begin{frame}[fragile]{11.2~容器\small{——顺序容器}}
	没有性能完美的容器,在选择顺序容器时,我们需要考虑以下几点:
	\vspace{0.3cm}
	\begin{itemize}[<+->]
		\item 如果需要高效的随机存取,不在乎插入和删除的效率,则使用vector;
		\item 如果需要大量的插入和删除元素,不关心随机存取的效率,则使用list;
		\item 如果需要随机存取,并且关心两端数据的插入和删除效率,则使用deque;
		\item 如果仅在读取输入的数据时在容器的中间位置插入元素,数据输入完毕之后仅需要随机访问,则可考虑在输入时将元素读入到一个list 容器中,然后对此容器使用sort函数排序,最后将排序后的list复制到一个vector容器中。
		\item 如果程序既需要随机访问又必须在容器的中间位置插入或删除元素,那么我们需要比较随机访问list 和在vector 中间插入或删除元素时移动元素的代价。
	\end{itemize}
%	\begin{redblock}
%		{注意}
%		如果无法确定某种应用应该采用哪种容器,则编写代码时尝试\alert{只使用vector和list容器都提供的操作},方便以后进行转换。
%	\end{redblock}
\end{frame}

\subsection{关联容器}
\begin{frame}[fragile]{11.2~容器\small{——关联容器}}
	\begin{block}{关联容器}
		不同于顺序容器,关联容器采用\alert{非线性}结构。通常情况下,关联容器是通过\alert{树结构}实现的,并通过\alert{关键字}来访问其元素。STL中主要的两个关联容器是:
		\begin{itemize}
			\item \textbf{set}
			\item \textbf{map}
		\end{itemize}
	\end{block}
\end{frame}

\begin{frame}[fragile]{11.2~容器\small{——关联容器}}
	\begin{block}
		{set}
		set中\alert{每个元素只包含一个关键字},与数学上的\alert{集合}类似,set\alert{不包含重复的元素},且它们都是有序的。
	\end{block}
	\begin{greenblock}
		{例11.1}
		统计输入的一组数字中不同数字的个数,并将它们排序输出
	\end{greenblock}
\vspace{-3mm}
	\begin{columns}
		\column{0.65\textwidth}
		\begin{blueblock}<2->{例11.1}
			\vspace{-3mm}\begin{lstlisting}[moreemph={T}]
set<int> counter; //创建一个关键字类型为int的空set对象
int number;
while (cin>>number) //输入数字
    counter.insert(number); //将输入的数字插入到set中
cout << "不同的数字的个数:" << counter.size() << endl;//获取元素个数
for (auto &i : counter) //遍历每个元素
    cout << i << " "; //输出每个元素
            \end{lstlisting}
		\end{blueblock}
		\column{0.3\textwidth}
		\begin{yellowblock}<2->{说明}
\scriptsize{
			$\bullet$ 向set插入元素时,如果已有,则将其抛弃;否则,按序将其插入。\\
			$\bullet$ 输入:1 8 4 2 0 1 4 3 5 4,\\输出:不同数字的个数:7 \\       0 1 2 3 4 5 8\\
			$\bullet$ 可见重复元素被排除,剩余按升序排列
}
		\end{yellowblock}
	\end{columns}
\end{frame}

\begin{frame}[fragile]{11.2~容器\small{——关联容器}}
	利用find函数查找、利用erase删除set中元素:
	\begin{columns}
		\column{0.65\textwidth}
		\begin{blueblock}{find查找set中元素}
			\begin{lstlisting}[moreemph={T}]
vector<int> v = { 1, 8, 4, 2, 0, 1, 4, 3, 5, 4, 7 };
set<int> s(v.begin(), v.end()); //利用vector创建set
auto it = s.find(0); //查找关键字为0的元素
            \end{lstlisting}
		\end{blueblock}
		\begin{blueblock}{erase删除set中元素}
			\begin{lstlisting}[moreemph={T}]
s.erase(it); //删除关键字为0的元素
s.erase(s.find(3), s.find(7)); //删除范围[3,7)内元素
for (auto &i : s)
    cout << i << " "; //打印输出:1 2 7 8
        \end{lstlisting}
		\end{blueblock}
		\column{0.3\textwidth}
		\begin{yellowblock}{说明}
			$\bullet$ 待查元素存在,则返回该元素的迭代器;否则返回尾后迭代器\\
			$\bullet$ erase成员的迭代器范围为左闭合区间\\
		\end{yellowblock}
		\begin{redblock}
			{注意}
			调用erase成员不影响与set中其它元素绑定的迭代器或引用。
		\end{redblock}
	\end{columns}
\end{frame}

\begin{frame}[fragile]{11.2~容器\small{——关联容器}}
	介绍map之前,首先了解一种标准库类型模板\textbf{pair}:
	\begin{block}{pair}
		pair定义在头文件utility中,包含两部分数据成员\\
	\end{block}
	\begin{blueblock}{使用pair}
		\begin{lstlisting}[moreemph={T}]
pair<int, int> p1; //保存两个int类型数据
pair<string, int> p2 = {"Hello", 0}; //列表初始化两个成员
auto p3 = make_pair("Hello", 1); //make_pair函数返回一个pair对象
cout << p2.first << p2.second << endl; //访问pair中数据成员
    \end{lstlisting}
	\end{blueblock}
	\begin{yellowblock}{说明}
		pair的两个数据成员是\alert{公有的},名字分别为first和second\\
	\end{yellowblock}
\end{frame}


\begin{frame}[fragile]{11.2~容器\small{——关联容器}}
	\begin{columns}[T]
		\column{0.65\textwidth}
		\begin{greenblock}{例11.2}
			统计输入的一组数字中\alert{每个数字出现的次数}
		\end{greenblock}
		\begin{blueblock}{使用map}
			\vspace{-3mm}\begin{lstlisting}[moreemph={T}]
map<int, int> counter; //创建map对象
int number;
while (cin >> number)
    ++counter[number];
for (auto &i : counter) //遍历map 中每个元素
    cout << i.first << ": " << i.second << endl;
输入:1 2 4 4 5 3 2 4 7 0 2
输出:
0:1
1: 1
2: 3
...
        \end{lstlisting}
		\end{blueblock}
		\column{0.3\textwidth}
		\begin{block}{map}
			map与set类似,都是有序容器。但map中的元素是pair类型,第一个成员为用于索引的\alert{关键字},第二个成员为与关键字相关的\alert{值}。
		\end{block}
		\only<1>{\begin{yellowblock}{说明}
				$\bullet$ 下标运算用来获取与关键字关联的值\\
				$\bullet$ 执行第四行代码时,如找到关键字对应元素则其值自增,否则以此关键字生成新的元素\\
			\end{yellowblock}}
		\only<2->{\begin{redblock}{注意}
				由于下标运算可能会插入新元素。因此,它只能作用于\alert{非const的map对象}\\
			\end{redblock}}
	\end{columns}
\end{frame}

\begin{frame}[fragile]{11.2~容器\small{——关联容器}}
	使用map:
	\begin{columns}[T]
		\column{0.65\textwidth}
		\begin{blueblock}{使用insert成员添加元素}
			\begin{lstlisting}[moreemph={T}]
counter.insert({ 3, 0 }); //C++11新特性
counter.insert(make_pair(3, 0));
        \end{lstlisting}
		\end{blueblock}
		\begin{blueblock}{检测插入是否成功}
			\begin{lstlisting}[moreemph={T}]
auto res=counter.insert(pair<int, int>(2,0));//自动推导res的类型
if (!res.second) //关键字2已经存在
    ++res.first->second; //关键字为2的元素的值自增
        \end{lstlisting}
		\end{blueblock}
		\column{0.3\textwidth}
		\begin{yellowblock}{说明}
			$\bullet$ insert函数返回一个pair对象,该对象的第一个成员为一个指向map中给定关键字的迭代器,第二个成员是一个bool值。给定关键字已存在则其值为false;否则为true
		\end{yellowblock}
	\end{columns}
\end{frame}

\begin{frame}[fragile]{11.2~容器\small{——关联容器}}
	提示:C++11新标准允许为关联容器进行列表初始化
	\begin{columns}[T]
		\column{0.65\textwidth}
		\begin{blueblock}{关联容器列表初始化}
			\begin{lstlisting}[moreemph={T}]
set<string> names = {"Kevin", "Lisha", "Mandy", "Rosieta"};
map<string, unsigned long long> contact = {
{"Kevin",15387120503}, {"Rosieta",15387120506}};
            \end{lstlisting}
		\end{blueblock}
		\column{0.3\textwidth}
		\begin{yellowblock}{说明}
			$\bullet$ 对于set,每个元素的类型即为关键字类型。对于map,每个元素的类型为一对花括号括起来的pair类型
		\end{yellowblock}
	\end{columns}
\end{frame}

\begin{frame}[fragile]{11.2~容器\small{——关联容器*}}
	
	\begin{greenblock}{multimap的使用范例}
		set和map中的关键字必须是唯一的。但有些情况下,比如我们存放电话簿时,同一个人可能有不同的手机号码,这时候应该怎么办?
	\end{greenblock}
\vspace{-3mm}
	\begin{columns}[T]
		\column{0.65\textwidth}
		\begin{blueblock}{使用multimap}
			\vspace{-3mm}\begin{lstlisting}[moreemph={T}]
multimap<string, unsigned long long> contact;
contact.insert({ "Kevin",15387120503 });
contact.insert({ "Kevin",15387120506 });
for (auto &i : contact)
    cout << i.first << ": " << i.second << endl;
auto entries = contact.count("Kevin");
auto it = contact.find("Kevin");
while (entries) {
    cout << it->second << endl; //打印电话号码
    ++it; //移动到下一个记录
    --entries; //计数器自减
}
            \end{lstlisting}
		\end{blueblock}
		\column{0.3\textwidth}
		\begin{yellowblock}{说明}
			$\bullet$ find语句将返回第一个关键字为Kevin的元素的迭代器\\
			$\bullet$ 利用count返回的值,不断递增it,直到Kevin的所有号码被打印\\
		\end{yellowblock}
	\end{columns}
\end{frame}

\subsection{高效使用容器}
\begin{frame}[fragile]{11.2~容器\small{——高效使用容器}}
	我们在编程中常常会面临容器的选择,而不合理的容器选择将会大大\alert{降低程序的效率}。本节将介绍一些常用的容器使用原则:
	\begin{itemize}
		\item 使用empty检查容器是否为空
		\item \alert{使用存放指针的容器}
		\item 使用算法和区间成员
		\item 使用reserve成员
		\item 使用有序的vector容器
		\item 正确使用map的insert和下标运算符
		\item 使用成员函数代替同名的算法
	\end{itemize}

\end{frame}

\begin{frame}[fragile]{11.2~容器\small{——高效使用容器*}}
\vspace{-3mm}
	\begin{yellowblock}{1.使用empty检查容器是否为空}
		empty和size成员均能用于检测容器是否为空,但是empty存在两个优势:
		\begin{itemize}
			\item empty总能保证\alert{常数时间内返回}
			\item empty是所有容器\alert{通用}的操作
		\end{itemize}
		而成员size不总是通用的,比如forforward\_list就没有提供size成员。\\
		因此建议使用成员empty来检查容器是否为空。
	\end{yellowblock}
\end{frame}

\begin{frame}[fragile]{11.2~容器\small{——高效使用容器}}
\vspace{-3mm}
	\begin{yellowblock}{2.使用存放指针的容器}
		如果容器中存放的对象是\alert{大对象}(占用较大内存空间),那么在操作容器过程中\alert{复制大对象}会使得程序付出很大的性能代价。此时我们可以考虑使用\alert{指针的容器}而不是对象的容器。
	\end{yellowblock}
	比如我们有以下“大”对象类
\end{frame}

\begin{frame}[fragile]{11.2~容器\small{——高效使用容器}}
\vspace{-3mm}
	\begin{columns}[T]
		\column{0.65\textwidth}
		\begin{blueblock}{定义“大”对象类}
			\vspace{-3mm}\begin{lstlisting}[moreemph={LargeData}]
struct LargeData {
    LargeData(int id): m_id(id){}
    int m_id;
    int m_arr[1000];
};
            \end{lstlisting}
		\end{blueblock}
		\begin{blueblock}{reverse“大”对象}
			\vspace{-3mm}\begin{lstlisting}[moreemph={LargeData}]
vector<LargeData*> vp;
vector<LargeData> vo;
for (int i = 0; i < 50000; i++){
    int n = rand() % 1000000; //生成一个随机数
    vp.emplace_back(new LargeData(n)); //尾插一个元素
    vo.emplace_back(n); //尾插一个元素
}
reverse(vo.begin(), vo.end()); //翻转对象
reverse(vp.begin(), vp.end()); //翻转指针
            \end{lstlisting}
		\end{blueblock}
		\column{0.3\textwidth}
		\begin{yellowblock}{说明}
			$\bullet$ 对vo的reverse操作会对对象进行复制,但对vp的reverse操作只涉及到指针的复制\\
			$\bullet$ 因此,对vp执行reverse操作会比对vo执行reverse操作花费的时间少很多。
		\end{yellowblock}
	\end{columns}
\end{frame}

\begin{frame}[fragile]{11.2~容器\small{——高效使用容器}}
	同时,为了方便地维护内存,\alert{避免内存泄漏问题},我们可以在容器中存放unique\_ptr
	% \begin{columns}[T]
	%     \column{0.65\textwidth}
	\begin{blueblock}{在容器中存放unique\_ptr}
		\begin{lstlisting}[moreemph={LargeData}]
vector<unique_ptr<LargeData>> vsp; //定义一个存放unique_ptr的vector
vp.push_back(make_unique<LargeData>(1)); //make_unique为C++14标准
vp.push_back(std::move(unique_ptr<LargeData>(new LargeData(2))));
vp.emplace_back(new LargeData(3));
            \end{lstlisting}
	\end{blueblock}
	% \column{0.3\textwidth}
	\begin{yellowblock}{说明}
		$\bullet$ 第三条语句使用move函数将一个临时对象移动到vp的尾部\\
		$\bullet$ 第四条语句使用emplace函数在vp的尾部直接构造一个对象。\\
	\end{yellowblock}
	% \end{columns}
	\pause
	\begin{redblock}{注意}
		对指针容器使用排序算法时,我们需要定义\alert{基于对象的比较函数}
	\end{redblock}
\end{frame}

\begin{frame}[fragile]{11.2~容器\small{——高效使用容器*}}
	\begin{yellowblock}{3.使用算法和区间成员}
		相比单元素遍历操作,使用区间成员的优势在于:1)\alert{更少的函数调用};2)\alert{更少的元素移动};3)\alert{更少的内存分配}
	\end{yellowblock}
\vspace{-3mm}
	\begin{columns}[T]
		\column{0.65\textwidth}
		\begin{blueblock}{插入元素:使用单元素遍历操作}
			\vspace{-3mm}\begin{lstlisting}[moreemph={T}]
int arr[] = { 1,2,4,10,5,4,1,8,20,30,15 };
vector<int> vi;
for (int i = 0; i < 7; i++)
    vi.push_back(arr[i]);
                \end{lstlisting}
		\end{blueblock}
		\begin{blueblock}{插入元素:使用区间成员}
			\vspace{-3mm}\begin{lstlisting}[moreemph={T}]
vi.assign(arr, arr + 7);
                \end{lstlisting}
		\end{blueblock}
		\column{0.3\textwidth}
		\only<1>{\begin{yellowblock}{说明}
				每次vi\alert{容量}小于需求时,vector先\alert{分配更大空间},然后\alert{移动}已有元素,最后\alert{添加}新元素\\
			\end{yellowblock}
			\begin{yellowblock}{}
				此处代码首先得到简化,并且减少了内存分配和数据移动的操作,提高了性能\\
			\end{yellowblock}}
		\only<2->{\begin{redblock}{注意}
				对于vector容器,成员\alert{capacity}指的是当前状态下,容器\alert{能容纳的}元素数目,而\alert{size}指的是当前容器中\alert{实际的}元素数目。
			\end{redblock}}
	\end{columns}
\end{frame}

\begin{frame}[fragile]{11.2~容器\small{——高效使用容器*}}
\vspace{-3mm}
	\begin{yellowblock}{4.使用reserve成员}
		对于vector 容器,如果\alert{预先知道数据需要的空间大小},可以利用\textbf{reserve成员}\alert{预先分配空间},这样会避免重新分配空间和移动已有元素产生的代价。
	\end{yellowblock}
\vspace{-3mm}
	\begin{columns}[T]
		\column{0.65\textwidth}
		\begin{blueblock}{插入元素:使用单元素遍历操作}
			\begin{lstlisting}[moreemph={T}]
vector<int> vi;
cout << "预留前,容量:"<<vi.capacity()<<"大小:"
  <<vi.size()<<endl;
vi.reserve(1000);
cout << "预留后,容量:"<<vi.capacity()<<"大小:"
  <<vi.size()<<endl;
                    \end{lstlisting}
		\end{blueblock}
		\begin{blueblock}{运行结果}
			输出:预留前,容量:0,大小:0\\
			输出:预留后,容量:1000,大小:0
		\end{blueblock}
		\column{0.3\textwidth}
		\begin{redblock}{注意}
			使用reserve只是重新分配内存空间,改变它的容量,但不会对vector产生resize行为,因此容器中的\alert{内容是不变的}
		\end{redblock}
	\end{columns}
\end{frame}

\begin{frame}[fragile]{11.2~容器\small{——高效使用容器*}}
\vspace{-3mm}
	\begin{yellowblock}{5.使用有序的vector容器}
		如果我们的操作是分阶段的,如一系列插入操作->查询操作,那么我们可以:
		\begin{enumerate}
			\item 使用\alert{有序关联容器}完成插入
			\item 使用关联容器创建\alert{有序vector}
			\item 使用vector进行查询
		\end{enumerate}
	\end{yellowblock}
\vspace{-3mm}
	\begin{columns}[T]
		\column{0.65\textwidth}
		\begin{blueblock}{使用有序的vector容器}
			\begin{lstlisting}[moreemph={T}]
multiset<int> s; //利用multset存放有序元素
int number;
while (cin >> number) //插入元素
    s.insert(number);
vector<int> v(s.begin(), s.end()); //创建有序vector
if (binary_search(v.begin(), v.end(), 10)) //二分查找
    cout << "10 is found" << endl;
else cout << "10 is not found" << endl;
            \end{lstlisting}
		\end{blueblock}
		\column{0.3\textwidth}
		\begin{blueblock}{运行结果}
			输入:10 20 10 30 15 20 10\\
			输出:10 is found
		\end{blueblock}
	\end{columns}
\end{frame}

\begin{frame}[fragile]{11.2~容器\small{——高效使用容器*}}
\vspace{-3mm}
	\begin{yellowblock}{6.正确使用map的insert和下标运算符}
		对于map来说,其成员insert和下标运算符有着不同的功能:
		\begin{itemize}
			\item 使用下标运算符意味着可能插入新的元素或覆盖已有元素的值
			\item insert专用于插入,不会覆盖已有元素
			\item at成员则只对元素进行访问
		\end{itemize}
	\end{yellowblock}
	如果不在意下标运算符是否会插入新的元素,则可尽情使用。比如:
\vspace{-3mm}
	\begin{columns}[T]
		\column{0.65\textwidth}
		\begin{blueblock}{前例11.2~统计数字出现的次数}
			\begin{lstlisting}[moreemph={T}]
map<int, int> counter;
int number;
while (cin >> number)
    ++counter[number];
for (auto &i : counter)
    cout << i.first << ": " << i.second << endl;
            \end{lstlisting}
		\end{blueblock}
		\column{0.3\textwidth}
		\begin{blueblock}{运行结果}
			输入:
			1 2 4 4 5 3 2 4 7 0 2\\
			输出:\\
			0: 1\\
			1: 1\\
			2: 3\\
			...
		\end{blueblock}
	\end{columns}
\end{frame}

\begin{frame}[fragile]{11.2~容器\small{——高效使用容器*}}
\vspace{-3mm}
	\begin{yellowblock}{7.使用成员函数代替同名的算法}
		有些容器的成员函数名和STL中算法的名字相同,它们都实现某种特定的功能。通常情况下,\alert{成员函数的效率要好于全局算法}。
	\end{yellowblock}
	\begin{columns}[T]
		\column{0.65\textwidth}
		\begin{blueblock}{查找——全局函数和成员函数}
			\begin{lstlisting}[moreemph={T}]
vector<int> v = { 3, 7, 3, 11, 3, 3, 2 };
set<int> s(v.begin(), v.end());
auto it1 = find(s.begin(), s.end(), 10); //查找速度慢
auto it2 = s.find(10); //查找速度快
            \end{lstlisting}
		\end{blueblock}
		\column{0.3\textwidth}
		\begin{yellowblock}{说明}
			$\bullet$ 全局find函数查找时为\alert{依次比较},为线性复杂度\\
			$\bullet$ set成员find会\alert{利用set的有序性快速查找},为对数复杂度
		\end{yellowblock}
	\end{columns}
\end{frame}

\section{泛型算法}
\begin{frame}[fragile]{11.3~泛型算法}
	标准库提供了可以用于不同容器的泛型算法。它们有一致的结构,大多数算法都接受一个范围迭代器,对此范围内的元素进行处理。算法并不需要了解所处理容器的类型。\\
	本节我们将介绍算法的框架和使用方法,算法详细介绍可参考在线手册:http://zh.cppreference.com/。
\end{frame}

\subsection{算法概述}
\begin{frame}[fragile]{11.3~泛型算法\small{——算法概述}}
	\begin{block}{标准库算法}
		依据算法对元素的访问方式,标准库算法主要有三大类:
		\begin{itemize}
			\item 只读型
			\item 写入型
			\item 重排型
		\end{itemize}
		大多数算法都定义在头文件algorithm中,基本语法格式见书中附录C。
	\end{block}
\end{frame}

\begin{frame}[fragile]{11.3~泛型算法\small{——算法概述}}
	\begin{block}{只读型算法}
		只是读取迭代器范围内的元素,不会改变元素的内容
	\end{block}
	\begin{columns}[T]
		\column{0.65\textwidth}
		\begin{blueblock}{只读算法案例1——find}
			\begin{lstlisting}[moreemph={T}]
vector<int> v = { 3, 7, 3, 11, 3, 3, 2 };
auto it = find(v.begin(), v.end(), 10);
cout << "10 is " << (it != v.end() ? "found" : "not found");
            \end{lstlisting}
		\end{blueblock}
		\column{0.3\textwidth}
		\begin{block}{find函数}
			遍历给定范围内的元素是否存在一个特定值
		\end{block}
	\end{columns}

\begin{yellowblock}{说明}
	$\bullet$ 前两个参数为迭代器范围,第三个参数为待搜索值\\
	$\bullet$ 它从开始位置\alert{依次}将每个元素与给定值比较\\
	$\bullet$ 如找到,返回\alert{第一个}与给定值相等的元素的迭代器\\
	$\bullet$ 否则返回\alert{第二个参数}表示搜索失败
\end{yellowblock}
\end{frame}

\begin{frame}[fragile]{11.3~泛型算法\small{——算法概述}}
	\begin{columns}[T]
		\column{0.65\textwidth}
		\begin{blueblock}{只读算法案例2——accumulate}
			\begin{lstlisting}[moreemph={T}]
int sum = accumulate(v.begin(), v.end(), 0);
            \end{lstlisting}
		\end{blueblock}
		\begin{blueblock}{只读算法案例2——accumulate}
			\begin{lstlisting}[moreemph={T}]
vector<string> vs = { "Hello ","world" };
string s1 = accumulate(vs.begin(), vs.end(), string()); //正确
string s2 = accumulate(vs.begin(), vs.end(), ""); //错误
            \end{lstlisting}
		\end{blueblock}
		\column{0.3\textwidth}
		\begin{block}{accumulate函数}
			计算特定范围内元素的和
		\end{block}
		\only<1>{\begin{yellowblock}{说明}
				$\bullet$ 第三个参数表示和的初始值\\
				$\bullet$ 初始值决定了返回值的类型\\
				$\bullet$ 下方第三条语句中的第三个参数类型为const char*。由于\alert{const char*没有定义+运算},因此产生编译错误\\
			\end{yellowblock}}
		\only<2->{\begin{redblock}{注意}
				$\bullet$ 如果元素的类型为string,那么将把范围内所有的string连接起来。\\
				$\bullet$ 其中第三个参数类型必须是string类型,不能是字符串常量
			\end{redblock}}
	\end{columns}
\end{frame}

%!此处估计要有图
\begin{frame}[fragile]{11.3~泛型算法\small{——算法概述}}
\vspace{-3mm}
	\begin{block}{写入型算法}
		将元素的值写入到容器中
	\end{block}
\vspace{-3mm}
	\begin{columns}[T]
		\column{0.65\textwidth}
		\begin{blueblock}{写入型算法案例——fill}
			\vspace{-3mm}\begin{lstlisting}[moreemph={T}]
vector<int> v1(10),v2(15);
fill(v1.begin(), v1.end(), 1); //将容器v1中所有元素重置为1
copy(v1.begin(), v1.end(), v2.begin()); //将v1中的元素复制到v2中
            \end{lstlisting}
		\end{blueblock}
		\column{0.3\textwidth}
		\begin{block}
			{fill函数}
			将\alert{给定值}写入到\alert{指定范围}
		\end{block}
		\begin{block}
			{copy函数}
			将\alert{给定范围内元素}依次复制到\alert{第三个参数指定的起始位置}
		\end{block}
	\end{columns}
	\begin{yellowblock}{说明}
		$\bullet$ fill函数前两个参数为迭代器范围(目的序列),第三个参数为写入值\\
		$\bullet$ copy函数前两个参数表示输入范围,第三个迭代器表示目的序列的起始位置\\
	\end{yellowblock}
\end{frame}

\begin{frame}[fragile]{11.3~泛型算法\small{——算法概述}}
\vspace{-3mm}
	\begin{block}{重排型算法}
		重新排列容器中的元素顺序
	\end{block}
\vspace{-3mm}
	\begin{columns}[T]
		\column{0.65\textwidth}
		\begin{blueblock}{重排型算法案例1——sort}
			\begin{lstlisting}[moreemph={T}]
vector<int> v = { 3, 7, 3, 11, 3, 3, 2 };
sort(v.begin(), v.end()); //升序排序
            \end{lstlisting}
		\end{blueblock}
		\begin{blueblock}{重排型算法案例2——stable\_sort}
			\begin{lstlisting}[moreemph={T}]
stable_sort(v.begin(), v.end());
            \end{lstlisting}
		\end{blueblock}
		\column{0.3\textwidth}
		\begin{block}
			{sort函数}
			使输入序列中的元素有序,默认的元素比较方式为<运算符
		\end{block}
		\begin{block}
			{stable\_sort函数}
			与sort函数作用相同,但会\alert{保持相等元素的相对位置}
		\end{block}
	\end{columns}
	\begin{redblock}{注意}
		数值相等的元素的相对位置在使用sort排序后可能会改变\\
	\end{redblock}
\end{frame}

\begin{frame}[fragile]{11.3~泛型算法\small{——算法概述}}
\vspace{-3mm}
	\begin{redblock}{注意}
		如果容器元素是用户自定义类型,则需要提供元素的<运算符\\
	\end{redblock}
\vspace{-3mm}
	\begin{columns}[T]
		\column{0.65\textwidth}
		\begin{blueblock}{重排型算法案例3——自定义<运算符}
			\vspace{-3mm}\begin{lstlisting}[moreemph={LargeData}]
struct LargeData {
    bool operator<(const LargeData rhs) {
        return m_id < rhs.m_id; //比较对象的id
    }//其它成员与11.2.4节相同
};            \end{lstlisting}
		\end{blueblock}
		\begin{blueblock}{重排型算法案例3——sort}
			\begin{lstlisting}[moreemph={LargeData}]
vector<LargeData> vo;
for (int i = 0; i < 50000; i++)
    vo.emplace_back(rand() % 1000000);
sort(vo.begin(), vo.end()); //按元素id的升序排序
            \end{lstlisting}
		\end{blueblock}
		\column{0.3\textwidth}
		\begin{yellowblock}
			{说明}
			$\bullet$ \alert{自定义类型LargeData}有了定义的<运算符之后,方可使用sort进行排序
		\end{yellowblock}
	\end{columns}
\end{frame}

\subsection{向算法传递函数}
\begin{frame}[fragile]{11.3~泛型算法\small{——向算法传递函数}}
	为了提高程序的效率,我们在11.2.4节中建议使用指针容器代替对象容器:
	\begin{blueblock}{指针容器代替对象容器}
		\vspace{-3mm}\begin{lstlisting}[moreemph={LargeData}]
        vector<LargeData*> vp;
    \end{lstlisting}
	\end{blueblock}
	由于sort算法将会按照\alert{指针大小排序}而不会对\alert{指针指向的对象}进行排序,所以我们需要\alert{定义自己的比较方式}。\\
\pause
	sort算法的第二版本的第三个参数接收一个\textbf{二元谓词},即自定义比较方式。
	\begin{block}
		{谓词}
		谓词是一个可以调用的表达式,其返回的结果能用于\alert{条件测试}。\\
		标准库算法使用的谓词有两种:
		\begin{itemize}
			\item 一元谓词(unary predicate):只接受一个参数
			\item 二元谓词(binary predicate):接受两个参数
		\end{itemize}
	\end{block}
\pause
	向算法传递可调用对象的三种方式:\textbf{使用函数}、\textbf{使用函数对象}和\textbf{使用lambda表达式}
\end{frame}

\begin{frame}[fragile]{11.3~泛型算法\small{——向算法传递函数}}
	我们定义下面一个函数,该函数接受两个LargeData类型指针,比较的是指针指向的对象的id:\\
	\begin{columns}[T]
		\column{0.65\textwidth}
		\begin{blueblock}{使用函数——定义待传递函数}
			\begin{lstlisting}[moreemph={LargeData}]
bool Less(const LargeData *a, const LargeData *b) {
    return a->m_id < b->m_id;
}
            \end{lstlisting}
		\end{blueblock}
		\begin{blueblock}{使用函数——传递函数}
			\begin{lstlisting}[moreemph={T}]
sort(vp.begin(), vp.end(), Less);
            \end{lstlisting}
		\end{blueblock}
		\column{0.3\textwidth}
		\begin{yellowblock}
			{说明}
			$\bullet$ 左面代码执行完以后,vp中的元素将会按照id升序排列\\
			$\bullet$ 当sort算法需要比较两个元素时,便会调用Less函数\\
		\end{yellowblock}
	\end{columns}
\end{frame}

\begin{frame}[fragile]{11.3~泛型算法\small{——向算法传递函数}}
	我们还可以向算法传递一个函数对象\\
	\begin{columns}[T]
		\column{0.65\textwidth}
		\begin{blueblock}{使用函数对象——定义类型}
			\begin{lstlisting}[moreemph={LargeData}]
struct Compare {
    bool operator()(const LargeData *a, const LargeData *b) {
        return a->m_id < b->m_id;
    }
};
            \end{lstlisting}
		\end{blueblock}
		\begin{blueblock}{使用函数对象——传递对象}
			\begin{lstlisting}[moreemph={T}]
sort(vp.begin(), vp.end(), Compare());
            \end{lstlisting}
		\end{blueblock}
		\column{0.3\textwidth}
		\begin{yellowblock}
			{说明}
			$\bullet$ 上方代码为Compare类定义一个函数调用运算符,形参和功能与Less函数一样\\
			$\bullet$ 下方sort函数调用中的第三个实参为通过Compare\alert{默认构造函数}创建的一个函数对象\\
			$\bullet$ 函数对象可以\alert{保存调用时的状态},相比于普通函数更为灵活\\
		\end{yellowblock}
	\end{columns}
\end{frame}

\begin{frame}[fragile]{11.3~泛型算法\small{——向算法传递函数*}}
	函数对象可以\alert{保存调用时的状态}。相比于普通函数,函数对象更加灵活,能够完成函数不能完成的任务。下面我们通过Checker类和find\_if算法来查找容器中第n个元素:\\
	\begin{columns}[T]
		\column{0.65\textwidth}
		\begin{blueblock}{使用函数对象——查找第n个元素}
			\begin{lstlisting}[moreemph={T}]
struct Checker{
    int m_cnt = 0, m_nth;
    Checker(int n) :m_nth(n) {}//初始化设定值
    bool operator()(int) { return ++m_cnt == m_nth; }
};
            \end{lstlisting}
		\end{blueblock}
		\begin{blueblock}{使用函数对象——查找第n个元素}
			\begin{lstlisting}[moreemph={T}]
vector<int> v = { 3, 7, 3, 11, 3, 3, 2 };
auto i=find_if(v.begin(),v.end(),Checker(4));//返回第4个元素的迭代器
            \end{lstlisting}
		\end{blueblock}
		\column{0.3\textwidth}
		\begin{yellowblock}
			{说明}
			$\bullet$ 两个数据成员分别用来计数(m\_cnt)和保存设定值(m\_nth)\\
			$\bullet$ 每一次调用Checker对象,其计数器就会自增,当增加到设置值时,返回真\\
			$\bullet$ 当调用返回真时,find\_if返回指向当前元素的迭代器\\
		\end{yellowblock}
	\end{columns}
\end{frame}

\begin{frame}[fragile]{11.3~泛型算法\small{——向算法传递函数}}
	一个lambda表达式为一个\alert{可调用的代码单元},因此也可以向算法传递一个lambda表达式:\\
	% \begin{columns}[T]
	% 	\column{0.65\textwidth}
	\begin{blueblock}{使用lambda表达式}
		\begin{lstlisting}[moreemph={T,LargeData}]
sort(vp.begin(), vp.end(),
  [](const LargeData *a, const LargeData *b) {return a->m_id < b->m_id;});
            \end{lstlisting}
	\end{blueblock}
	% \column{0.3\textwidth}
	\begin{yellowblock}
		{说明}
		$\bullet$ lambda表达式捕获列表为空\\
		$\bullet$ 函数形参为两个指针类型\\
		$\bullet$ 函数体与Less函数和Compare函数调用运算符一样\\
	\end{yellowblock}
	% \end{columns}
\pause
	和上面两种方法相比,使用lambda表达式:
	\begin{itemize}
		\item 它\alert{不需要额外定义一个函数或一个函数对象类}
		\item 可以利用捕获列表\alert{访问外围对象}
	\end{itemize}
	如果可调用对象的\alert{操作比较简单}且\alert{只在局部使用},lambda表达式是最佳选择。
\end{frame}

\subsection{参数绑定}
\begin{frame}[fragile]{11.3~泛型算法\small{——参数绑定*}}
	有些标准库算法\alert{只接受一个包含一个参数的调用对象},但有时候我们想要\alert{传递给算法的函数包含两个参数}。比如,我们希望通过filter函数将容器中小于一个给定值的元素设置为0
	\begin{blueblock}{\texttt{filter}函数}
		\vspace{-3mm}\begin{lstlisting}[moreemph={T}]
void filter(int &a, int n) {
    a = a < n ? 0 : a;
}
            \end{lstlisting}\vspace{-3mm}
	\end{blueblock}
	同时我们希望通过for\_each算法来遍历元素,但是该算法第三个参数接受只包含一个参数的可调用对象。此时我们可以通过lambda表达式来实现:

	\begin{columns}[T]
		\column{0.65\textwidth}
		\begin{blueblock}{\texttt{filter}函数}
			\vspace{-3mm}\vspace{-3mm}\begin{lstlisting}[moreemph={T}]
vector<int> vi = { 3, 7, 1, 11, 3, 3, 2 };
int n = 3;
for_each(vi.begin(), vi.end(), [n](int &i) { i = (i < n ? 0 : i);});
            \end{lstlisting}
		\end{blueblock}
		\column{0.3\textwidth}
		\only<1>{\begin{yellowblock}
				{说明}
				通过lambda捕获列表,我们可以从\alert{外部}设定n的值\\
			\end{yellowblock}}
		\only<2->{\begin{greenblock}
				{问题}
				如果坚持使用filter呢?\\
			\end{greenblock}}
	\end{columns}
\end{frame}

\begin{frame}[fragile]{11.3~泛型算法\small{——参数绑定*}}
	如果我们依然坚持用filter函数代替lambda表达式,我们可以使用标准库bind
	\begin{columns}[T]
		\column{0.65\textwidth}
		\begin{blueblock}{\texttt{bind}使用格式}
			\begin{lstlisting}[moreemph={T}]
auto newFun = bind (fun, arg_list);
            \end{lstlisting}
		\end{blueblock}
		\column{0.3\textwidth}
		\begin{block}
			{bind函数}
			bind函数接受一个可调用对象,生成一个新的可调用对象来仿造原调用对象的参数列表。\\
		\end{block}
	\end{columns}
	\begin{yellowblock}
		{说明}
		$\bullet$ \texttt{fun}是一个已定义的调用对象, \texttt{newFun}是\texttt{fun}的仿造者\\
		$\bullet$ \texttt{arg\_list}是\texttt{fun}的参数列表\\
		$\bullet$ \texttt{arg\_list}可能包含一些名为\texttt{\_n}的参数,他们是\texttt{newFun}的参数,\texttt{n}的值表示在\texttt{newFun}参数列表中的位置。\\
		$\bullet$ 当我们调用\texttt{newFun}时,\texttt{newFun}会调用\texttt{fun},并把\texttt{arg\_list}中的参数传递给\texttt{fun}。\\
	\end{yellowblock}
\end{frame}

\begin{frame}[fragile]{11.3~泛型算法\small{——参数绑定*}}
	下面我们根据filter函数仿造一个新的调用对象uf:
	\begin{columns}[T]
		\column{0.65\textwidth}
		\begin{blueblock}{使用bind——根据filter仿造新对象uf}
			\begin{lstlisting}[moreemph={T}]
auto uf = bind(filter, std::placeholders::_1, n);
            \end{lstlisting}
		\end{blueblock}
		\begin{blueblock}{使用bind——调用for\_each}
			\begin{lstlisting}[moreemph={T}]
for_each(vi.begin(), vi.end(), uf);
        \end{lstlisting}
		\end{blueblock}
		\column{0.3\textwidth}
		\begin{yellowblock}
			{说明}
			$\bullet$ 仿函数uf包含一个参数\_1\\
			$\bullet$ 调用uf时,将参数\_1和参数n传递给filter函数\\
			$\bullet$ 调用for\_each 时,给定的元素传递给uf,uf将这个元素和n 传递给filter\\
		\end{yellowblock}
	\end{columns}
\end{frame}

\begin{frame}[fragile]{11.3~泛型算法\small{——参数绑定*}}
	默认情况下,bind函数中\alert{不是占位符的参数}将以\alert{拷贝}的方式传递给可调用对象。如果需要\alert{传递引用},可以使用标准库函数\textbf{ref}:
	\begin{columns}[T]
		\column{0.65\textwidth}
		\begin{blueblock}{使用bind——传递引用}
			\begin{lstlisting}[moreemph={T}]
void sum(int a, int &s){
s += a;
}
            \end{lstlisting}
		\end{blueblock}
		\begin{blueblock}{使用bind——传递引用}
			\begin{lstlisting}[moreemph={T}]
int s = 0; //保存累加和
for_each(vi.begin(), vi.end(), bind(sum, std::placeholders::_1, ref(s)));
            \end{lstlisting}
		\end{blueblock}
		\column{0.3\textwidth}
		\begin{yellowblock}
			{说明}
			$\bullet$ sum将第一个函数的值累加到与形参s绑定的实参中\\
			$\bullet$ ref函数返回一个包含s的引用的对象\\
		\end{yellowblock}
		\begin{redblock}
			{提示}
			标准库提供类似ref的cref函数,其返回包含\alert{const引用类型}的对象
		\end{redblock}
	\end{columns}
\end{frame}

\subsection{使用function}
\begin{frame}[fragile]{11.3~泛型算法——\small{使用function}}
	前述内容中用于LargeData对象比较的调用对象,例如函数、函数对象、bind函数创建的对象等等虽然使用方式不同但都具有\alert{相同的调用形式}:
	\begin{columns}[T]
		\column{0.65\textwidth}
		\begin{blueblock}{用于比较的可调用对象的形式}
			\begin{lstlisting}[moreemph={LargeData}]
bool(LargeData*, LargeData*)
            \end{lstlisting}
		\end{blueblock}
		%         \only<2->{\begin{blueblock}{使用function2}
		% 			\begin{lstlisting}[moreemph={T}]
		% LargeData a(0), b(1);
		% if (f1(&a, &b)) {/*...*/}
		% if (f2(&a, &b)) {/*...*/}
		% if (f3(&a, &b)) {/*...*/}
		% if (f4(&a, &b)) {/*...*/}
		%             \end{lstlisting}
		% 		\end{blueblock}}
		\column{0.3\textwidth}
		\begin{yellowblock}
			{说明}
			该\alert{调用形式}是一个函数类型,接受两个LargeData指针类型的形参,返回值为bool\\
		\end{yellowblock}
		% \only<2->{\begin{yellowblock}
		% 	{说明}
		% 	$\bullet$ 使用新创建的function对象对上面4种不同的调用对象实现统一的使用方式
		% \end{yellowblock}}
	\end{columns}
\end{frame}

\begin{frame}[fragile]{11.3~泛型算法——\small{使用function}}
	我们可以将前述不同表现形式的对象用\textbf{function类模板}统一起来:
	\begin{columns}[T]
		\column{0.65\textwidth}
		\begin{blueblock}{定义function对象}

			\vspace{-2mm}\begin{lstlisting}[moreemph={T}]
using CallType = bool(LargeData*, LargeData*);
function<CallType> f1 = Less; //函数
function<CallType> f2 = Compare(); //函数对象
function<CallType> f3 = [](const LargeData *a, const LargeData *b) {return a->m_id < b->m_id; }; //lambda
function<CallType> f4 = //bind函数
bind(Less, std::placeholders::_2, std::placeholders::_1);
            \end{lstlisting}\vspace{-2mm}
		\end{blueblock}
		\begin{blueblock}{使用function对象}
            \vspace{-2mm}
			\begin{lstlisting}[moreemph={T}]
LargeData a(0), b(1);
if (f1(&a, &b)) {/*...*/}
if (f2(&a, &b)) {/*...*/}
if (f3(&a, &b)) {/*...*/}
if (f4(&a, &b)) {/*...*/}
                \end{lstlisting}
            \vspace{-2mm}
		\end{blueblock}
		\column{0.3\textwidth}
		\begin{block}
			{funcion}
			\alert{通用多态函数封装器},其实例能存储、复制及调用任何\alert{可调用 (Callable) 目标}
		\end{block}
		\begin{yellowblock}
			{说明}
			使用CallType类型统一了各种形式的调用\\
		\end{yellowblock}
	\end{columns}
\end{frame}

\begin{frame}[c]{~}
	\begin{center}
		\huge{本章结束}
	\end{center}
\end{frame}

\end{document}
